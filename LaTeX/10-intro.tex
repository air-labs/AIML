\documentclass[24pt,pdf,hyperref={unicode}]{beamer}
\usepackage[utf8]{inputenc}
\usepackage[russian]{babel}
\usepackage{aiml}
\usepackage{framed}


\newcommand{\citate}[3]
{
\begin{framed}
{
\fontfamily{Liberation Serif}\selectfont
#1
}
\end{framed}
\begin{flushright}
(#2, {\it #3})
\end{flushright}
}

\begin{document}

\section{Рождение теории}

\begin{frame}
\bio{Turing}{Алан Тьюринг}{1912--1954}{}
\end{frame}

\begin{frame}
\citate{
If at each stage the motion of a machine (in the sense of § 1) is completely
determined by the configuration, we shall call the machine an "automatic machine" (or $a$-machine).

For some purposes we might use machines (choice machines or
c-manhines) whose motion is onty partially determined by the configuration
... When such a machine
reaches one of these ambiguous configurations, it cannot go on until some
arbitrary choice has been made by an external operator ... In this
paper I deal only with automatic machines, and will therefore often omit
the prefix a-.
}{Alan Turing}{On computable numbers, with an application to the Entscheidungsproblem}
\end{frame}

 

\begin{frame}
\citate
{
The behaviour of the \alert<2->{computer} at any moment is determined by the symbols which \alert<3->{he} is observing, and \alert<4>{his "state of mind"} at that moment. We may suppose that there is a bound $B$ to the number of symbols or squares which the computer can observe at one moment. If \alert<4>{he wishes} to observe more, he must use successive observations. We will also suppose that the number of \alert<4>{states of mind} which need be taken into account is finite.
}{Alan Turing}{On computable numbers, with an application to the Entscheidungsproblem}

\end{frame}

\begin{frame}
\citate
{
...a game which we call the ``imitation game''. It is played (by) a man (A), a woman (B), and an interrogator (C). The interrogator stays in a room apart front the other two. The object of the game for the interrogator is to determine which of the other two is the man and which is the woman. He knows them by labels X and Y. The interrogator is allowed to put questions to A and B thus:\\[0.2cm]

C: Will X please tell me the length of his or her hair?\\[0.2cm]

Now suppose X is actually A, then A must answer:\\[0.2cm]

``My hair is shingled, and the longest strands are about nine inches long.''
}{Alan Turing}{Computing machinery and intelligence}

\end{frame}

\begin{frame}

\citate
{
We now ask the question, ``What will happen when a machine takes the part of A in this game?'' Will the interrogator decide wrongly? These questions replace our original, ``Can machines think?''
}{Alan Turing}{Computing machinery and intelligence}

\end{frame}

\begin{frame}
\begin{columns}
\column{0.5\textwidth}
\bio{McCulloch}{Warren McCulloch}{1898-1969}{}
\column{0.5\textwidth}
\bio{Pitts}{Walter Pitts}{1923-1969}{}
\end{columns}
\end{frame}

\begin{frame}
\bio{Rosenblatt}{Frank Rosenblatt}{1928-1971}{}
\end{frame}

\begin{frame}
\bio{Barricelli}{Nils Aal Bariccelli}{1912-1993}{}
\end{frame}

\begin{frame}
\citate
{
We propose that a 2 month, 10 man study of artificial intelligence be carried out during the summer of 1956 at Dartmouth College in Hanover, New Hampshire. The study is to proceed on the basis of the conjecture that every aspect of learning or any other feature of intelligence can in principle be so precisely described that a machine can be made to simulate it. An attempt will be made to find how to make machines use language, form abstractions and concepts, solve kinds of problems now reserved for humans, and improve themselves. We think that a significant advance can be made in one or more of these problems if a carefully selected group of scientists work on it together for a summer.
}{McCarthy et al.}{Proposal for the project}
\end{frame}

\section{Золотой век}

\end{document}
