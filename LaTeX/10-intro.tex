\documentclass[24pt,pdf,hyperref={unicode}]{beamer}
\usepackage[utf8]{inputenc}
\usepackage[russian]{babel}
\usepackage{graphics}
\usepackage{amssymb}
\usepackage{xstring}
\usepackage{multirow}
\usepackage{framed}

\newcommand{\bio}[3]
{
\begin{center}
\includegraphics[width=4cm]{#1.jpg}

#2

#3
\end{center}
}

\newcommand{\citate}[3]
{
\begin{framed}
{
\fontfamily{Liberation Serif}\selectfont
#1
}
\end{framed}
\begin{flushright}
(#2, {\it #3})
\end{flushright}
}

\begin{document}

\begin{frame}
\bio{Turing}{Алан Тьюринг}{1912--1954}
\end{frame}

\begin{frame}
\citate{
If at each stage the motion of a machine (in the sense of § 1) is completely
determined by the configuration, we shall call the machine an "automatic machine" (or $a$-machine).

For some purposes we might use machines (choice machines or
c-manhines) whose motion is onty partially determined by the configuration
... When such a machine
reaches one of these ambiguous configurations, it cannot go on until some
arbitrary choice has been made by an external operator ... In this
paper I deal only with automatic machines, and will therefore often omit
the prefix a-.
}{Alan Turing}{On computable numbers, with an application to the Entscheidungsproblem}
\end{frame}

 

\begin{frame}
\citate
{
The behaviour of the \alert<2->{computer} at any moment is determined by the symbols which \alert<3->{he} is observing, and \alert<4>{his "state of mind"} at that moment. We may suppose that there is a bound $B$ to the number of symbols or squares which the computer can observe at one moment. If \alert<4>{he wishes} to observe more, he must use successive observations. We will also suppose that the number of \alert<4>{states of mind} which need be taken into account is finite.
}{Alan Turing}{On computable numbers, with an application to the Entscheidungsproblem}

\end{frame}




\end{document}
