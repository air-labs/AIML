\documentclass[24pt,pdf,hyperref={unicode}]{beamer}
\usepackage[utf8]{inputenc}
\usepackage{aiml}

\begin{document}



\section{Задача регрессии}

\begin{frame}\frametitle{Задача регрессии}
{\bf Дано:}
\begin{tabular}{p{4cm} p{6cm}}
 $\mathcal{X}=(X_1,\ldots,X_k)$ & экспериментальные условия, $X_i\in\mathbb{R}^n$\\[0.1cm]
 $\mathcal{A}=(A_1,\ldots,A_k)$ & измеренные значения, $A_i\in\mathbb{R}^m$\\[0.1cm]
 $(\mathcal{X},\mathcal{A})$ & экспериментальная база \\[0.1cm]
 $P$ & вектор параметров \\[0.1cm]
 $F(P,X)$ & функция регрессии \\[0.1cm]
 $\sum_{i=1}^{k} (F(P,X_i)-A_i)^2 $ & среднеквадратичное отклонение\\
 \end{tabular}\\[1cm]
 {\bf Найти:}
 параметры $P$ такие, что $\sum_{i=1}^{k} (F(P,X_i)-A_i)^2 \rightarrow \min$
\end{frame}

\begin{frame}\frametitle{Функция Tanh}
\begin{tikzpicture}[x=1cm,y=2cm]
\draw[->,thick] (-5,0) -- (5,0) node[below] {$x$};
\draw[->,thick] (0,-1.2) -- (0,1.2) node[left] {$tanh(x)$};
\draw[thin] (1,0) node[below]{1} -- (1,0.78);
\draw[thin] (2,0) node[below]{2} -- (2,0.95);
\draw[red, ultra thick] plot file {Plots/Tanh.txt};

\end{tikzpicture}
\end{frame}

\end{document}