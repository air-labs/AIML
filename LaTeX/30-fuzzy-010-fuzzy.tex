\documentclass[24pt,pdf,hyperref={unicode}]{beamer}
\usepackage[utf8]{inputenc}
\usepackage{aiml}
\usepackage{pgfplots}



\begin{document}

\begin{frame}
\bio{Zadeh}{Лотфи Задэ}{1921--н.д.}{Создатель нечеткой математики}
\end{frame}


\begin{frame}\frametitle{Нечеткие логические связки}
$$
\begin{array}{c p{1cm} l}
\uncover<+->{x,y\in\{0,1\}} 
&&
\uncover<+->{u,v\in[0,1]}\\
\uncover<+->{\begin{array}{c|c}
x & \neg x=\overline{x}\\
\hline
0 & 1 \\
1 & 0 \\
\end{array}}
&&
\uncover<+->{\neg u=(1-u)}
\\
\uncover<+->{\begin{array}{c c|c c}
x & y & x\wedge y & x\vee y \\
\hline
0 & 0 & 0 & 0 \\
0 & 1 & 0 & 1 \\
1 & 0 & 0 & 1 \\
1 & 1 & 1 & 1 \\
\end{array}}
&&
\uncover<+->{\begin{array}{l}
u\fwedge v=\max(u,v)\\
u\fvee v=\max(u,v)\\
\end{array}}
\\
\end{array}
$$
\end{frame}

\begin{frame}\frametitle{Нечеткие логические связки}
$$
\begin{array}{c p{0.1cm} c}

x\vee y && u\fvee v=\max(u,v) \\
x\wedge y && u\fwedge v=\max(u,v) \\[0.3cm]

\uncover<+->{
x\vee y=y\vee x && \max(u,v)=\max(v,u)
}\\[0.3cm]

\uncover<+->{
x\wedge y=y\wedge x  && \min(u,v)=\min(u,v)
}\\[0.3cm]

\uncover<+->{
x\vee (y\vee z)=(x\vee y)\vee z
&&
\max(u,\max(v,w))=\max(\max(u,v),w)
}\\[0.3cm]

\uncover<+->{
x\wedge(y\wedge z)=(x\wedge y)\wedge z
&&
\min(u,\min(v,w))=\min(\min(u,v),w)
}\\[0.3cm]

\uncover<+->{
\overline{x\vee y}=\overline{x}\wedge\overline{y} 
&&
1-\max(u,v)=\min(1-u,1-v)
}\\[0.3cm]

\uncover<+->{
\overline{x\wedge y}=\overline{x}\vee\overline{y}
&&
1-\min(u,v)=\max(1-u,1-v) 
}\\
\end{array}
$$
\end{frame}

\begin{frame}\frametitle{Нечеткие логические связки}
$$
\begin{array}{c p{0.1cm} c}
x\vee y &&  u\fvee v=u+v-uv \\
x\wedge y && u\fwedge v=uv \\[0.3cm]

\uncover<+->{
x\vee y=y\wedge x && u+v-uv=v+u-vu
}\\[0.3cm]

\uncover<+->{
x\wedge y=y\vee x  && uv=vu
}\\[0.3cm]

\uncover<+->{x\vee (y\vee z)=(x\vee y)\vee z
&&
\begin{array}{c}
u+(v+w-vw)-u(v+u-vw)=\\
=u+v+w-uv-uw-vw+uvw\\
\end{array}
}\\[0.3cm]

\uncover<+->{
x\wedge(y\wedge z)=(x\wedge y)\wedge z && u(vw)=(uv)w
}\\[0.3cm]


\uncover<+->{
\overline{x\vee y}=\overline{x}\wedge\overline{y} 
&&
\begin{array}{c}
1-(u+v-vw)=1-u-v+vw=\\
=(1-u)(1-v)
\end{array}
}\\[0.4cm]

\uncover<+->{
\overline{x\wedge y}=\overline{x}\vee\overline{y}
&&
\begin{array}{c}
(1-u)+(1-v)-(1-u)(1-v)=\\
=1-u+1-v-1+u+v+uv= \\
=1-uv
\end{array}
}
\end{array}
$$
\end{frame}

\begin{frame}\frametitle{Нормы и конормы}
Функции $T,S:[0,1]\times[0,1]\rightarrow[0,1]$ называют нормой и конормой, если они:

\begin{enumerate}
\item монотонны;
\item ассоциативны;
\item коммутативны;
\item связаны соотношениями де-Моргана $1-T(u,v)=S(1-u,1-v)$ и $1-S(u,y)=T(1-u,1-v0)$;
\item удовлетворяют граничным условиям $T(0,0)=T(0,1)=T(1,0)=0$, $T(1,1)=1$, $S(1,1)=S(0,1)=T(1,0)=1$, $S(0,0)=0$
\end{enumerate}
\end{frame}



\begin{frame}\frametitle{Нечеткие множества}


\end{frame}

\begin{frame}\frametitle{Объединение и пересечение}
\end{frame}

\begin{frame}\frametitle{Объединение и пересечение}
\end{frame}

\end{document}