\documentclass[24pt,pdf,hyperref={unicode},aspectratio=169]{beamer}
\usepackage[utf8]{inputenc}
\usepackage{aiml}
\usepackage{pgfplots}
\usepackage{etex}
\usepackage{aiml}
\usepackage{ctable}
\usepackage{xifthen}
\usepackage{xparse}

\newcommand{\ImageSizeAlone}{0.6cm}
\newcommand{\ImageSizeFuzzy}{0.4cm}

\newcommand{\fuzzyimage}[2]{
\ifthenelse{\isempty{#2}}
{
\ifmmode{\begin{array}{c}\includegraphics[width=\ImageSizeAlone]{HOMM5/#1.jpg}\end{array}}
\else{\includegraphics[width=\ImageSizeAlone]{HOMM5/#1.jpg}}
\fi
}
{
\dfrac{#2}{\includegraphics[width=\ImageSizeFuzzy]{HOMM5/#1.jpg}}
}
}

\newcommand{\da}{}
\DeclareDocumentCommand\da{O{}}{\fuzzyimage{dark}{#1}}
\newcommand{\de}{}
\DeclareDocumentCommand\de{O{}}{\fuzzyimage{destructive}{#1}}

\newcommand{\pe}{}
\DeclareDocumentCommand\pe{O{}}{\fuzzyimage{peasant}{#1}}
\newcommand{\wt}{}
\DeclareDocumentCommand\wt{O{}}{\fuzzyimage{wraith}{#1}}
\newcommand{\ga}{}
\DeclareDocumentCommand\ga{O{}}{\fuzzyimage{garg}{#1}}
\newcommand{\un}{}
\DeclareDocumentCommand\un{O{}}{\fuzzyimage{unicorn}{#1}}

\newcommand{\ar}{}
\DeclareDocumentCommand\ar{O{}}{\fuzzyimage{archer}{#1}}
\newcommand{\fo}{}
\DeclareDocumentCommand\fo{O{}}{\fuzzyimage{footman}{#1}}
\newcommand{\mo}{}
\DeclareDocumentCommand\mo{O{}}{\fuzzyimage{monk}{#1}}
\newcommand{\sk}{}
\DeclareDocumentCommand\sk{O{}}{\fuzzyimage{skeleton}{#1}}
\newcommand{\gh}{}
\DeclareDocumentCommand\gh{O{}}{\fuzzyimage{ghost}{#1}}
\newcommand{\lc}{}
\DeclareDocumentCommand\lc{O{}}{\fuzzyimage{lich}{#1}}

\newcommand{\aca}{}
\DeclareDocumentCommand\aca{O{}}{\fuzzyimage{academy}{#1}}
\newcommand{\hel}{}
\DeclareDocumentCommand\hel{O{}}{\fuzzyimage{inferno}{#1}}
\newcommand{\nec}{}
\DeclareDocumentCommand\nec{O{}}{\fuzzyimage{necropolis}{#1}}


\newcommand{\vc}{\centering\arraybackslash}






\begin{document}

\section{Развитие логических систем}

\begin{frame}\frametitle{Логика высших порядков}

\uncover<+->{
$P$ является биекцией из $A$ в $B$
$$
\forall x \forall y \forall z\ P(x,y)\wedge P(x,z)\rightarrow Eq(y,z)
$$
$$
\forall x \exists y\ A(x)\wedge B(y)\wedge P(x,y)
$$
$$
\forall y \exists x\ A(x)\wedge B(y)\wedge P(x,y)
$$
$$
\forall x \forall y\ \forall z P(x,z) \wedge P(y,z) \rightarrow Eq(x,y)
$$}

\uncover<+->{
Равномощность $A$ и $B$:

$$
\exists P \left[ \forall x \forall y \forall z\ P(x,y)\wedge P(x,z)\rightarrow Eq(y,z) \right]\wedge\ldots
$$}


\end{frame}

\begin{frame}\frametitle{Принцип доказательства от противного}

\begin{itemize}
\item<+-> Необходимо доказать, что $\exists x P(x)$
\item<+-> Предположим, что $\forall x \neg P(x)$
\item<+-> Придем к противоречию
\item<+-> Следовательно, $\exists x P(x)$
\end{itemize}
\uncover<+->{
\begin{center}
Но чему равен $x$?
\end{center}
}
\end{frame}

\begin{frame}\frametitle{Модальная логика}

\uncover<+->{
Модальные операторы:

\begin{itemize}
\item $KA$ -- $А$ известно
\item $\diamond A$ -- $A$ возможно 
\end{itemize}}

\uncover<+->{
\begin{tabular}{l l l}
A1 & Принцип объективности знания & $KA\rightarrow A$ \\
A2 & Дистрибутивность знания и конъюнкции & $K(A\wedge B)\rightarrow KA\wedge KB$ \\
A3 & Принцип познаваемости мира & $A\rightarrow \diamond K A$ \\
\end{tabular}
}

\begin{itemize}
\item<+-> Предположим, $A\wedge\neg KA$ 
\item<+-> По A3, $\diamond K (A\wedge\neg KA)$ 
\item<+-> По A2, $\diamond ( KA \wedge K(\neg KA) )$ 
\item<+-> По А1, $\diamond ( KA \wedge \neg KA) $
\item<+-> Противоречие. Все уже познано.
\end{itemize}
\end{frame}

\section{Нечеткие логические связки}

\begin{frame}
 \bio
 {Zadeh}{Lotfi Zadeh}
 {Fuzzy sets (1965)}
\end{frame}


\begin{frame}\frametitle{Нечеткие логические связки}
$$
\begin{array}{c p{1cm} l}
\uncover<+->{x,y\in\{0,1\}} 
&&
\uncover<+->{u,v\in[0,1]}\\[0.5cm]
\uncover<+->{\begin{array}{c|c}
x & \neg x=\overline{x}\\
\hline
0 & 1 \\
1 & 0 \\
\end{array}}
&&
\uncover<+->{\neg u=(1-u)}
\\[1cm]
\uncover<+->{\begin{array}{c c|c c}
x & y & x\wedge y & x\vee y \\
\hline
0 & 0 & 0 & 0 \\
0 & 1 & 0 & 1 \\
1 & 0 & 0 & 1 \\
1 & 1 & 1 & 1 \\
\end{array}}
&&
\uncover<+->{\begin{array}{l}
u\fwedge v=\min(u,v)\\
u\fvee v=\max(u,v)\\
\end{array}}
\\
\end{array}
$$
\end{frame}

\begin{frame}\frametitle{Нечеткие логические связки}
$$
\begin{array}{c p{0.1cm} c}

x\vee y && u\fvee v=\max(u,v) \\
x\wedge y && u\fwedge v=\max(u,v) \\[0.3cm]

\uncover<+->{
x\vee y=y\vee x && \max(u,v)=\max(v,u)
}\\[0.3cm]

\uncover<+->{
x\wedge y=y\wedge x  && \min(u,v)=\min(u,v)
}\\[0.3cm]

\uncover<+->{
x\vee (y\vee z)=(x\vee y)\vee z
&&
\max(u,\max(v,w))=\max(\max(u,v),w)
}\\[0.3cm]

\uncover<+->{
x\wedge(y\wedge z)=(x\wedge y)\wedge z
&&
\min(u,\min(v,w))=\min(\min(u,v),w)
}\\[0.3cm]

\uncover<+->{
\overline{x\vee y}=\overline{x}\wedge\overline{y} 
&&
1-\max(u,v)=\min(1-u,1-v)
}\\[0.3cm]

\uncover<+->{
\overline{x\wedge y}=\overline{x}\vee\overline{y}
&&
1-\min(u,v)=\max(1-u,1-v) 
}\\
\end{array}
$$
\end{frame}

\begin{frame}\frametitle{Нечеткие логические связки}
$$
\uncover<+->{
\begin{array}{c p{0.1cm} c}
x\vee y &&  u\fvee v=u+v-uv \\
x\wedge y && u\fwedge v=uv \\[0.3cm]
}

\uncover<+->{
x\vee y=y\vee x && u+v-uv=v+u-vu
}\\[0.3cm]

\uncover<+->{
x\wedge y=y\wedge x  && uv=vu
}\\[0.3cm]

\uncover<+->{x\vee (y\vee z)=(x\vee y)\vee z
&&
\begin{array}{c}
u+(v+w-vw)-u(v+u-vw)=\\
=u+v+w-uv-uw-vw+uvw\\
\end{array}
}\\[0.3cm]

\uncover<+->{
x\wedge(y\wedge z)=(x\wedge y)\wedge z && u(vw)=(uv)w
}\\[0.3cm]


\uncover<+->{
\overline{x\vee y}=\overline{x}\wedge\overline{y} 
&&
\begin{array}{c}
1-(u+v-vw)=1-u-v+vw=\\
=(1-u)(1-v)
\end{array}
}\\[0.4cm]

\uncover<+->{
\overline{x\wedge y}=\overline{x}\vee\overline{y}
&&
\begin{array}{c}
(1-u)+(1-v)-(1-u)(1-v)=\\
=1-u+1-v-1+u+v+uv= \\
=1-uv
\end{array}
}
\end{array}
$$
\end{frame}

\begin{frame}\frametitle{Нормы и конормы}
Функции $T,S:[0,1]\times[0,1]\rightarrow[0,1]$ называют нормой и конормой, если они:

\begin{enumerate}
\item монотонны;
\item ассоциативны;
\item коммутативны;
\item связаны соотношениями де Моргана $1-T(u,v)=S(1-u,1-v)$ и $1-S(u,y)=T(1-u,1-v)$;
\item удовлетворяют граничным условиям $T(0,0)=T(0,1)=T(1,0)=0$, $T(1,1)=1$, $S(1,1)=S(0,1)=T(1,0)=1$, $S(0,0)=0$
\end{enumerate}
\end{frame}

\section{Нечеткие множества}

\begin{frame}\frametitle{Нечеткие множества}
$$
\begin{array}{c p{0.1cm} c}
\uncover<1->{
\mathbb{A},\ A\subset\mathbb{A},\ a\in A
&&
\mathbb{M},\ M\fsubset\mathbb{M},\ m\fin M
}
\\[0.3cm]
\uncover<2->{
(a,A)\overset{\in}{\rightarrow}\{0,1\} 
&&
(m,M)\overset{\fin}{\rightarrow}[0,1]
}
\\
\uncover<2->{
&&
\mu_M(m),\ \mu_M:\mathbb{M}\rightarrow[0,1]
}
\\[0.3cm]
\uncover<3->{
A=\{a_1,a_2,\ldots,a_n\}
&&
M=\left(\fuz{m_1}+\fuz{m_2}+\ldots+\fuz{m_n}\right)
}
\\[0.3cm]
\uncover<4->{
B\subset A\Leftrightarrow \forall b\ (b \in B\rightarrow b\in A)
&&
N\fsubset M\Leftrightarrow \forall m\ \mu_N(m)\le\mu_M(m)
}
\\[0.3cm]
\uncover<5->{
c\in A\cap B\Leftrightarrow c\in A\wedge c\in B
&&
\mu_{M\fcap N}(m)=\mu_M(m)\fwedge\mu_N(m)=
}
\\
\uncover<5->{
&& T(\mu_M(m),\mu_N(m))
}
\\[0.3cm]
\uncover<6->{
c\in A\cup B\Leftrightarrow c\in A\vee c\in B
&&
\mu_{M\fcup N}(m)=\mu_M(m)\fvee\mu_N(m)=\\
&& S(\mu_M(m),\mu_N(m))
}
\\[0.3cm]
\end{array}
$$
\end{frame}


\begin{frame}\frametitle{Объединение и пересечение}
\begin{tikzpicture}[x=1.8cm,y=3cm]
\draw[->,thick] (0,0.01) -- (5.5,0.01) node[below] {$x$};
\draw[->,thick] (0,0) -- (0,1.2) node[left] {$\mu$};


\node at(3,1.1) {$B\approx 3$};
\node at(2,1.1) {$A\approx 2$};
\draw[red,thick] plot file {Plots/2.txt};
\draw[blue,thick] plot file {Plots/3.txt};


\only<1>
{
\draw[dotted] (3,1) -- (0,1) node[left] {$1$};
\draw[dotted] (2,1) -- (2,0) node[below] {$2$};
\draw[dotted] (3,1) -- (3,0) node[below] {$3$};
}


\only<2>
{
\fill[fill=teal,opacity=0.2] plot file {Plots/2_cap1_3.txt};
\node at (2.5,0.2) {$A\fcup B,\ \max$};
}

\only<3>
{
\fill[fill=teal,opacity=0.2] plot file {Plots/2_cup1_3.txt};
\node at (2.5,0.2) {$A\fcap B,\ \min$};
}

\only<4>
{
\fill[fill=teal,opacity=0.2] plot file {Plots/2_cap2_3.txt};
\node at (2.5,0.2) {$\begin{array}{c}A\fcup B\\a+b-ab\end{array}$};

}

\only<5>
{
\fill[fill=teal,opacity=0.2] plot file {Plots/2_cup2_3.txt};
\node at (2.5,0.2) {$A\fcap B,\ ab$};
}


%\fill[fill=orange] plot file {Plots/A_cap1_B.txt};
%\fill[fill=orange] plot file {Plots/A_cap1_B.txt};

\end{tikzpicture}
\end{frame}


\section{Отношения и отображения}

\begin{frame}\frametitle{Отношения и отображения}
\begin{columns}
\column{0.5\textwidth}
$
A=\{a_1,a_2, a_3\}$, 

$B=\{b_1,b_2,b_3\}$, 

$C=\{c_1,c_2\}$

$$
\rho\subset A \times B=
\begin{array}{l|l l l}
    & b_1 & b_2 & b_3 \\
\hline
a_1 &  0  &  1 & 0 \\
a_2 &  1  &  0 & 1 \\
a_3 &  0  &  0 & 0 \\
\end{array}
$$

$$
\sigma\subset B \times C =
\begin{array}{l|l l l}
    & c_1 & c_2 \\
\hline
b_1 &  1  &  0 \\
b_2 &  0  &  1 \\
b_3 &  0  &  1 \\
\end{array}
$$

\column{0.5\textwidth}

\only<2>{
$\rho(a)=\{ b\ :\ (a,b)\in\rho \}$\\[0.5cm]

$\rho(a_1)=\{ b_2 \}$

$\rho(a_2)=\{ b_1, b_3 \}$

$\rho(a_3)=\emptyset$\\[0.5cm]

$\sigma(b_1)=c_1$

$\sigma(b_2)=c_2$

$\sigma(b_3)=c_2$\\[0.5cm]

$\rho \ne \rho:A\rightarrow B$

$\sigma = \sigma:B\rightarrow C$
}

\only<3>{

$\sigma^{-1}=\{(c,b)\ :\ (b,c)\in\sigma\}$\\[0.5cm]

$\sigma^{-1}(c_1)=b_1$

$\sigma^{-1}(c_2)=\{b_2,b_3\}$\\[0.5cm]

$\rho^{-1}(b_1)=a_2$

$\rho^{-1}(b_2)=a_1$

$\rho^{-1}(b_3)=a_2$\\[0.5cm]

$\rho^{-1} = \rho^{-1}:B\rightarrow A$

$\sigma^{-1} \ne \sigma^{-1} :C\rightarrow B$
}

\only<4>{

$$
\begin{array}{l l}
\rho\circ\sigma=\{ & (a,c)\ :\ \exists b \\
& (a,b)\in\rho,\ (b,c)\in\sigma\}
\end{array}
$$

$$
\rho\circ\sigma=
\begin{array}{l|l l }
    & c_1 & c_2 \\
\hline
a_1 &  0  &  1  \\
a_2 &  1  &  1  \\
a_3 &  0  &  0  \\
\end{array}
$$


}
\end{columns}
\end{frame}



\begin{frame}[t]\frametitle{Нечеткие отношения}

$$
A\subset\mathbb{A},\ \rho\subset\mathbb{A}\times\mathbb{B}
$$
\only<2->{$$
\rho(A)=\{ b\in\mathbb{B}\ :\ \exists a\in A, (a,b)\in\rho\}
$$}
\only<3->{$$
= \bigcup_{a\in\mathbb{A}}\{\underbrace{b,\ a\in A\wedge (a,b)\in\rho}_{\rho(A/a)\ne\rho(a)}\}
$$}
\only<4-7>{$$
b\in\rho(A/a)\Leftrightarrow a\in A\ \wedge\ (a,b)\in\rho
$$}
\only<5->{$$
M \fsubset \mathbb{M},\ \sigma\fsubset\mathbb{M}\times\mathbb{N}
$$}
\only<6-7>{$$
n\fin\sigma(M/m) = m\fin M\ \fwedge\ (m,n)\fin\sigma
$$}
\only<7-10>{$$
\mu_{\sigma(M/m)}(n)= T\left(\mu_{M}(m),\mu_{\sigma}(m,n)\right)
$$}
\only<8-10>{$$
\sigma(M)=\foperation{\bigcup_{m\in\mathbb{M}}}\sigma(M/m)
$$}
\only<9->{$$
\mu_{\sigma(M)}(n)=\underset{m\in\mathbb{M}}{S}\left[T\left(\mu_{M}(m),\mu_{\sigma}(m,n)\right)\right]
$$
\only<10->{$$
\mu_{\sigma(M)}(n)=\max_{m\in\mathbb{M}}\left[\mu_{M}(m)\mu_{\sigma}(m,n)\right]
$$}
}

\end{frame}

\section{Пример нечетких отношений}

\begin{frame}\frametitle{Нечеткие отношения}
\begin{columns}[t]
\column{0.55\textwidth}
\uncover<2->{$$
\mathbb{M}=\left\{\da,\de\right\}
$$
$$
\mathbb{C}=\left\{\pe,\ga,\wt,\un\right\}
$$
$$
\rho\fsubset\mathbb{C}\times\mathbb{M}
$$
}
\uncover<3->{
$$
\begin{array}{ >\vc m{0.7cm}| >\vc m{0.7cm} >\vc m{0.7cm}}
$\rho$   & \de & \da \\
\hline
\vskip 1pt
\pe & 0.8 & 0.8 \\
\wt & 0.8 & 0.2 \\
\ga & 0.2 & 0.8 \\
\un & 0.2 & 0.2 
\end{array}
$$
}
\column{0.45\textwidth}
\only<4->{$$
\only<4-11,13->{\mu_{\rho(C)}(m)=\max_{c\in\mathbb{C}}\left[\mu_C(c) \mu_\rho(c,m)\right]}
\only<12>{\mu_{\rho(C)}(m)=\underset{c\in\mathbb{C}}{S}\left[\mu_C(c) \mu_\rho(c,m)\right]}
$$}
\only<5-7>{$
\rho\left(\wt[1]\right)=\left(\de[0.8]+\da[0.2]\right)
$}
\only<6-7>{$
\rho\left(\pe[1]\right)=\left(\de[0.8]+\da[0.8]\right)
$}
\only<7>{
$
\rho\left(\un[1]\right)=\left(\de[0.2]+\da[0.2]\right)
$
}
\only<8-10>{
$
\rho\left(\un[0.7]+\ga[0.3]\right)=
$
}
\only<9-10>{
$
\left(\begin{array}{l}
\de[\max(0.7\cdot0.2,0.3\cdot0.2)]\\
\\
\da[\max(0.7\cdot0.2,0.3\cdot 0.8)]
\end{array}\right)
$
}
\only<10>{
$
=\left(\de[0.16]+\da[0.24]\right)
$
}

\only<11>{
$
\rho\left(\un[0.8]+\ga[0.2]\right)=
$


$
\left(\begin{array}{l}
\de[\max(0.8\cdot0.2,0.2\cdot0.2)]\\
\\
\da[\max(0.8\cdot0.2,0.2\cdot 0.8)]
\end{array}\right)
$

$
=\left(\de[0.16]+\da[0.16]\right)
$
}

\only<12>{
$
\rho\left(\un[0.8]+\ga[0.2]\right)=
$


$
\left(\begin{array}{l}
\de[\begin{array}{l}0.8\cdot0.2+0.2\cdot0.2-\\-0.8\cdot0.2\cdot0.2\cdot0.2\end{array}]\\
\\
\da[\begin{array}{l}0.8\cdot0.2+0.2\cdot 0.8-\\-0.8\cdot0.2\cdot0.8\cdot0.2\end{array}]
\end{array}\right)
$

$
=\left(\de[0.1936]+\da[0.2944]\right)
$
}


\only<13-15>{
$
\rho\left(\ga[0.4]+\wt[0.5]\right)=
$
}
\only<14-15>{
$
\left(\begin{array}{l}
\de[\max(0.4\cdot0.2,0.5\cdot0.8)]\\
\\
\da[\max(0.4\cdot0.8,0.5\cdot 0.2)]
\end{array}\right)
$
}
\only<15>{
$
=\left(\de[0.4]+\da[0.32]\right)
$
}


\end{columns}
\end{frame}

\section{Композиция нечетких отношений}


\begin{frame}[t]\frametitle{Композиция нечетких отношений}

\renewcommand{\ImageSizeAlone}{0.4cm}

\begin{columns}[t]
\column{0.5\textwidth}
$$
\mathbb{M}
=\left\{\da,\de\right\}
$$
$$
\mathbb{C}=\left\{\pe,\ga,\wt,\un\right\}
$$
$$
\mathbb{H}
=\left\{\aca,\hel,\nec\right\}
$$
$$
\rho\fsubset\mathbb{C}\times\mathbb{M}
$$
$$
\begin{array}{ >\vc m{0.7cm}| >\vc m{0.7cm} >\vc m{0.7cm}}
$\rho$   & \de & \da \\
\hline
\vskip 1pt
\pe & 0.8 & 0.8 \\
\wt & 0.8 & 0.2 \\
\ga & 0.2 & 0.8 \\
\un & 0.2 & 0.2 
\end{array}
$$

\column{0.6\textwidth}
$$
\sigma\fsubset\mathbb{M}\times\mathbb{H}
$$
$$
\begin{array}{ >\vc m{0.7cm}| >\vc m{0.7cm} >\vc m{0.7cm} >\vc m{0.7cm} }
$\sigma$   & \aca & \hel & \nec \\
\hline
\vskip 1pt
\de & 0.5 & 0.7 & 0.3 \\
\da & 0.5 & 0.3 & 0.7 \\
\end{array}
$$
$$
\tau=\rho\circ\sigma
,\ \tau\fsubset\mathbb{C}\times\mathbb{H}
$$
$$
\begin{array}{ >\vc m{0.7cm}| >\vc m{0.7cm} >\vc m{0.7cm} >\vc m{0.7cm}}
$\tau$   & \aca & \hel & \nec \\
\hline
\vskip 1pt
\pe &  &  &  \\
\wt &  &  &  \\
\ga &  &  &  \\
\un &  &  &  \\ 
\end{array}
$$
\end{columns}
\end{frame}

\renewcommand{\ImageSizeAlone}{0.6cm}

\begin{frame}[t]\frametitle{Композиция нечетких отношений}
\begin{columns}[t]
\column{0.5\textwidth}
$$
\begin{array}{ >\vc m{0.7cm}| >\vc m{0.7cm} >\vc m{0.7cm}}
$\rho$   & \de & \da \\
\hline
\vskip 1pt
\pe & 0.8 & 0.8 \\
\wt & 0.8 & 0.2 \\
\ga & 0.2 & 0.8 \\
\un & 0.2 & 0.2 
\end{array}
$$

$$
\begin{array}{ >\vc m{0.7cm}| >\vc m{0.7cm} >\vc m{0.7cm} >\vc m{0.7cm} }
$\sigma$   & \aca & \hel & \nec \\
\hline
\vskip 1pt
\de & 0.5 & 0.7 & 0.3 \\
\da & 0.5 & 0.3 & 0.7 \\
\end{array}
$$
\column{0.5\textwidth}
$$
\tau=\rho\circ\sigma
$$
\only<2->{$$
\mu_\tau(c,h)=\max_{m\in\mathbb{M}}\left[\mu_\rho(c,m)\mu_\sigma(m,h)\right]
$$}
\only<3-4>{$
\mu_\tau\left(\wt,\hel\right)=
$ $
\max\left[0.8\cdot0.7,0.2\cdot0.3\right]=0.56
$\\[0.5cm]}
\only<4>{$
\mu_\tau\left(\wt,\nec\right)=
$ $
\max\left[0.8\cdot0.3,0.2\cdot0.7\right]=0.24
$}
\only<5->{$$
\begin{array}{ >\vc m{0.7cm}| >\vc m{0.7cm} >\vc m{0.7cm} >\vc m{0.7cm}}
$\tau$   & \aca & \hel & \nec \\
\hline
\vskip 1pt
\pe & \uncover<7->{0.4} & \uncover<8->{0.56} &  \uncover<9->{0.56} \\
\wt & \uncover<10->{0.4} & \uncover<10->{0.58} & \uncover<10->{0.24} \\
\ga & \uncover<11->{0.4} & \uncover<11->{0.24} & \uncover<11->{0.58} \\
\un & \uncover<12->{0.1} & \uncover<12->{0.14} & \uncover<12->{0.14} \\ 
\end{array}
$$
}
\end{columns}
\end{frame}



\end{document}