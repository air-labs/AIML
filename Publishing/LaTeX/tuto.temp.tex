\documentclass[24pt,pdf,hyperref={unicode},aspectratio=169]{beamer}
\usepackage[utf8]{inputenc}
\usepackage[russian]{babel}
\usepackage{graphics}
\usepackage{amssymb}
\usepackage{xstring}
\usepackage{multirow}
\usepackage{tikz}
\usepackage[all]{xy}


\newcommand{\dd}[2]{\frac{\partial #1}{\partial #2}}

\tikzstyle{neu}=[circle,fill=blue!50,minimum size=0.8cm]

\begin{document}
\section{Методы ускорения обучения}
\begin{frame}\frametitle{Ускорение обучения}

\begin{itemize}
\item<+-> Правильный выбор коэффициента
$$
W_{i+1}=W_i-\varepsilon \nabla f(W_i)
$$

(начать с $\varepsilon=1$)

\item<+-> Повторение алгоритма градиентного спуска на каждом примере (3-5 раз)

\item<+-> Правило момента

$$
\Delta{i} = W_{i}-W_{i-1}
$$
$$
W_{i+1} = W_i-\varepsilon \nabla f(W_i)+\alpha\Delta_i, \ \ \ 
\alpha\approx 0.1
$$
\item<+-> Стимуляция нейронов

$$
W_{i+1} = (1-\gamma)W_i-\varepsilon \nabla f(W_i)+\alpha\Delta_i,\ \ \ 
\gamma \approx 10^{-4}
$$

\item<+-> Адаптивный выбор $\varepsilon$?

\item<+-> Сопряженные градиенты, обучение на гессиане??
\end{itemize}


\end{frame}


\end{document}
\end{document}